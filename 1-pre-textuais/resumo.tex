Com o aumento do uso da tecnologia, ela passou a ser explorada em diversas áreas da sociedade, e uma delas é a educação. Assim, surgiram diversos softwares, como o Moodle e o Kahoot, com o intuito de auxiliar na forma de ensino atual, seja em sala de aula, como meios de se aplicar perguntas para a turma e colher suas respostas, seja em atividades extraclasse, como forma de se revisar conteúdos e aplicar questões.

Nesse contexto, dada a crescente popularidade do uso de chatbots nos últimos anos, observou-se a possibilidade de eles serem aplicados no ambiente acadêmico. Por conta da popularidade de aplicativos de mensagens, como o Telegram e o WhatsApp, os chatbots podem atuar como meios de facilitar o acesso dos alunos a recursos educacionais pré-cadastrados pelos professores das suas disciplinas.

É possível, ainda, ir além e trabalhar no chatbot uma abordagem de ensino adaptável às necessidades de cada aluno, o que permite reforçar o aprendizado de partes do conteúdo que não foram totalmente compreendidas em sala de aula, e também auxiliar na fixação e na revisão dos demais conteúdos.

Assim, neste estudo, é proposto o desenvolvimento de um chatbot no Telegram para ser aplicado no ensino superior, que será capaz de processar as mensagens dos alunos em linguagem natural e fornecer recursos pertinentes ao seu aprendizado.

Desse modo, espera-se construir um ambiente de aprendizado digital de fácil acesso com o intuito de auxiliar os alunos durante o processo de aprendizado, visando uma melhora na absorção do conteúdo durante as disciplinas da graduação.

\palavraschave{Chatbot. Sistemas adaptativos. Educação de ensino superior}