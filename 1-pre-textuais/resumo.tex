Com o aumento do uso da tecnologia, ela passou a ser explorada em diversas áreas da sociedade, e uma delas é a educação. Assim, surgiram diversos softwares, como o Moodle e o Kahoot, com o intuito de auxiliar na forma de ensino atual, seja em sala de aula, como meios de se aplicar perguntas para a turma e colher suas respostas, seja em atividades extraclasse, como forma de se revisar conteúdos e aplicar questões. Nesse contexto, dada a crescente popularidade do uso de Chatbots integrados à Modelos de Linguagem de Larga-Escala nos últimos anos, observou-se a possibilidade de eles serem aplicados no ambiente acadêmico, como uma ferramenta voltada a prestar assistência aos alunos durante o aprendizado. Assim, neste trabalho, é apresentado o desenvolvimento do Coruja, um Chatbot do Telegram integrado à Modelos de Linguagem de Larga-Escala, com o intuito de ser aplicado na educação de ensino superior, capaz de responder dúvidas dos alunos com base em uma base de conhecimento pré-cadastrada pelo professor sobre o conteúdo da disciplina por meio da técnica de Geração Aumentada de Recuperação, além de permitir também que o professor cadastre exercícios objetivos nos quais os alunos poderão praticar o conhecimento adquirido. Desse modo, espera-se que o Coruja seja um ambiente de aprendizado digital de fácil acesso voltado a auxiliar os alunos durante o processo de aprendizado no decorrer da graduação.

\palavraschave{Chatbot. Educação de ensino superior. Modelos de Linguagem de Larga-Escala}