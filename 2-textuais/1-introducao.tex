\chapter{Introdução}
\label{cap:introducao}

\section{Contextualização}
\label{sec:contextualizacao}

De acordo com \citeonline{briel2021}, agentes conversacionais são sistemas capazes de processar e responder linguagem natural. Eles evoluíram com o passar dos anos, indo desde um meio para passar no Teste de Turing até chatbots que possuem um objetivo utilitário.

Ainda para \citeonline{briel2021}, é possível realizar uma distinção entre agentes de domínio aberto e de domínio fechado, enquanto os de domínio aberto são capazes de manter uma conversa sobre diversos assuntos, os de domínio fechado são mais orientadas à tarefas e tendem a conseguir retornar informações mais aprofundadas dentro de um determinado escopo.

Chatbots podem ser definidos como programas de computadores feitos para simular a conversação humana na forma de voz ou texto, ou até mesmo ambos \apud{wong2022}{wales2008}. O primeiro chatbot foi criado em 1966 por Joseph Weizenbaum, e desde então, eles são usados no ensino de áreas como línguas estrangeiras \apud{wong2022}{wales2008}, psicologia \apud{wong2022}{fryer2006} e até mesmo como técnicas de entrevista \apud{wong2022}{tseng2018}.

De acordo com \citeonline{wong2022}, a maioria da pesquisa em chatbots como uma ferramenta de aprendizado é focada na simulação da capacidade de conversação humana, e estando ela limitada apenas a algumas disciplinas acadêmicas. Assim, dado os avanços e a popularidade que os chatbots vêm ganhando nos últimos anos, surge a necessidade de analisar o potencial deles como uma ferramenta de ensino numa variedade maior de disciplinas.

No Campus de Russas da Universidade Federal do Ceará, a disciplina de Cálculo Numérico, ministrada pelo professor orientador desse trabalho, tem tido sua metodologia de ensino-aprendizagem modificada nos últimos anos para incentivar uma maior autonomia por parte dos alunos. Assim, torna-se necessário criar meios tecnológicos de fácil acesso que deem suporte aos estudos individuais dos alunos.

Desse modo, observou-se a possibilidade de aplicar chatbots na disciplina de Cálculo Numérico, como uma tecnologia acessível e que visa auxiliar na absorção e consolidação do conteúdo ministrado na disciplina. É válido destacar que, apesar da motivação inicial do estudo vir de uma disciplina específica da universidade, a sua aplicabilidade não se restringe à ela, o que permite a aplicação desse trabalho em cursos que abordam outras áreas do conhecimento.

\section{Objetivo geral}
\label{sec:objetivo-geral}

O principal objetivo desse trabalho é propor o desenvolvimento de um chatbot capaz de entender as necessidades dos alunos por meio de processamento de linguagem natural e fornecer recursos pertinentes ao seu aprendizado.

\section{Objetivos específicos}
\label{sec:objetivos-especificos}

Os objetivos específicos desse trabalho são:

\begin{itemize}
	\item Identificar padrões existentes de desenvolvimento de chatbots, bem como tecnologias relacionadas que podem auxiliar na seu desenvolvimento;
    \item Definir o funcionamento do chatbot proposto, junto com as tecnologias que serão utilizados no seu desenvolvimento;
	\item Desenvolver o chatbot com base no que foi especificado;
	\item Aplicar o chatbot desenvolvido em alguma disciplina no Campus de Russas da Universidade Federal do Ceará, se possível, a disciplina de Cálculo Numérico.
\end{itemize}

\section{Organização do trabalho}
\label{sec:organizacao-trabalho}

Este trabalho está organizado da seguinte maneira. O Capítulo \ref{cap:introducao} traz a contextualização da área de pesquisa desse trabalho, bem como seus objetivos gerais e específicos. O Capítulo \ref{cap:trabalhos-relacionados} aborda uma série de trabalhos relacionados ao tema desse trabalho. O Capítulo \ref{cap:metodologia} apresenta a metodologia desse trabalho, as tecnologias escolhidas para o desenvolvimento do chatbot, e também como será o seu funcionamento. O Capítulo \ref{cap:resultados-esperados} cita os resultados esperados com a aplicação do chatbot proposto. O Capítulo \ref{cap:cronograma} contém o cronograma das etapas seguintes desse trabalho. Por fim, o Capítulo \ref{cap:conclusao} traz a conclusão desse trabalho.