\chapter{Resultados esperados}
\label{cap:resultados-esperados}

Mediante o exposto, devido à escolha do Telegram como interface do chatbot, espera-se desenvolver um ambiente de aprendizado online de fácil acesso para os alunos, aproveitando da visibilidade que a plataforma já possui, em comparação com o desenvolvimento de um site ou aplicativo que serviria como interface para uma aplicação similar à proposta neste trabalho.

Com o chatbot, almeja-se que tanto professores quanto alunos se beneficiem do seu uso em disciplinas.

Com relação aos alunos, o chatbot dará maior autonomia a eles durante o processo educacional, já que será possível obter facilmente recursos pertinentes ao conteúdo das disciplinas, o que poderia auxiliar na fixação do conteúdo em momentos extraclasse.

Como consequência, os professores também terão benefícios ao aplicar o chatbot em suas disciplinas, pois, a partir das interações com o chatbot, as dúvidas dos alunos seriam sanadas com os materiais cadastrados pelo professor, o que viria a diminuir a demanda pelo docente fora de sala de aula.